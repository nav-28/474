\documentclass[12pt]{exam}
\printanswers
\usepackage[utf8]{inputenc}
\usepackage[a4paper, total={6in,4in}]{geometry}
\usepackage{geometry}
\usepackage{amsmath}
\usepackage{amssymb}
\usepackage{tabto}
\usepackage{makecell}
\usepackage{multicol}
\usepackage{enumerate}
\geometry{left=2cm, right=2cm, top=2cm, bottom=2cm}

\title{
  Assignment 4
}

\author{Pranav Wadhwa}

\begin{document}
\maketitle
\noindent


\begin{questions}

  \question{} % q1
  Answer each part TRUE or FALSE

  \begin{multicols}{2}
  \begin{enumerate}[a.]

    \item $2^{\log^{2}n}\in \mathcal{O}(n^{k})$
    \item $n!\in 2^{\mathcal{O}(n\log n)}$
    \item $3^{n}\in \mathcal{O}(2^{n\log n})$
    \item $n^{k}\in o(2^{\log n})$
    \item $2^{n}\in o(n!)$
    \item $\frac{1}{n} \in o(1)$

  \end{enumerate}
\end{multicols}


\begin{solution}

  \begin{enumerate}[a.]
    \item $2^{\log^{2}n}\in \mathcal{O}(n^{k})$

          True\\
          Proof: $2^{\log^{2}n} = n$ and $n\in \mathcal{O}(n^{k})$
    \item $n!\in 2^{\mathcal{O}(n\log n)}$

          True

    \item $3^{n}\in \mathcal{O}(2^{n\log n})$

          False\\
          As $3^{n}$ grows much faster than $2^{n\log n}$
    \item $n^{k}\in o(2^{\log n})$
          True\\

    \item $2^{n}\in o(n!)$

    \item $\frac{1}{n} \in o(1)$

          True as $1/n$ is strictly smaller than 1 after $n>1$

  \end{enumerate}



\end{solution}


  \question{} % q2

  Show that P is closed under union, complement, concatenation, and star.


  \question{} % q3

  Show that NP is closed under union, concatenation, and star

  \question{} % q4

  We normally assume natural numbers are represented in binary, such that a number $n\in \mathcal{N}$ is represented by the string $b_{\lfloor\log n \rfloor}b_{\lfloor \log n \rfloor - 1} \dots b_{0}, b_{i} \in \{0,1\}$, and $n=\Sigma^{\lfloor \log n \rfloor}_{i=0}b_{i}2^{i}$.
  We could also write a number in unary, where a number $n\in \mathcal{N}$ is represented by $n$ consecutive 1s.
  The problem of factoring a number in binary is not known to be in P, but what if the number is given in unary?
  Prove your answer.


  \question{} % q5

  Suppose the number $k$ and the graph $G$ are given, and we want to know if there exists a subset $S$ of size $k$ from the vertices of $G$ such that there is no edge between them in $G$.
  Prove that this problem is NP-Complete.


  \question{} % q6

  Show the following languge is NP-Complete:
  \[DOUBLE-SAT=\{\langle \phi \rangle | \phi \text{ has at least 2 satisfying assignments }\}\]


  \question{} % q7

  Let $S$ be a set and let $C$ be a collection of subsets of $S$. A set $S'\subseteq S$ is called a set hitting set for $C$ if every subset in $C$ contains at least an element in $S'$. Let
  \[HITSET = \{\langle C,k \rangle | C \text{ has a hitting set of size } k\}\]

  Prove that $HITSET$ is NP-Complete.


  \question{} % q8

  \begin{enumerate}[a.]
    \item Prove that NP $=$ coNP iff there is an NP-Complete problem in coNP.
    \item Show that if coNP $\neq$ NP then P $\neq$ NP.

  \end{enumerate}


  \question{} % q9

  Show that P is closed under homomorphism iff P $=$ NP

  \question{} % q10

  Let $CNF_{k} = \{\langle \phi \rangle | \phi \text{ is a satisfiable cnf-formula where each variable appears in at most } k \text{ places }\}$.

  \begin{enumerate}[a.]
    \item Show that $CNF_{2} \in P$
    \item Show that $CNF_{3}$ is NP-complete
  \end{enumerate}


\end{questions}


\end{document}
