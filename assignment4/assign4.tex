\documentclass[12pt]{exam}
\printanswers
\usepackage[utf8]{inputenc}
\usepackage[a4paper, total={6in,4in}]{geometry}
\usepackage{geometry}
\usepackage{amsmath}
\usepackage{amssymb}
\usepackage{tabto}
\usepackage{makecell}
\usepackage{multicol}
\usepackage{enumerate}
\geometry{left=2cm, right=2cm, top=2cm, bottom=2cm}

\title{
  Assignment 4
}

\author{Pranav Wadhwa}

\begin{document}
\maketitle
\noindent


\begin{questions}

  \question{} % q1
  Answer each part TRUE or FALSE

  \begin{multicols}{2}
  \begin{enumerate}[a.]

    \item $2^{\log^{2}n}\in \mathcal{O}(n^{k})$
    \item $n!\in 2^{\mathcal{O}(n\log n)}$
    \item $3^{n}\in \mathcal{O}(2^{n\log n})$
    \item $n^{k}\in o(2^{\log n})$
    \item $2^{n}\in o(n!)$
    \item $\frac{1}{n} \in o(1)$

  \end{enumerate}
\end{multicols}


\begin{solution}

  \begin{enumerate}[a.]
    \item $2^{\log^{2}n}\in \mathcal{O}(n^{k})$

          True\\
          Proof: $2^{\log^{2}n} = n$ and $n\in \mathcal{O}(n^{k})$
    \item $n!\in 2^{\mathcal{O}(n\log n)}$

          True

    \item $3^{n}\in \mathcal{O}(2^{n\log n})$

          False\\
          As $3^{n}$ grows much faster than $2^{n\log n}$
    \item $n^{k}\in o(2^{\log n})$
          True\\

    \item $2^{n}\in o(n!)$
          True

    \item $\frac{1}{n} \in o(1)$

          True as $1/n$ is strictly smaller than 1 after $n>1$

  \end{enumerate}



\end{solution}


  \question{} % q2

  Show that P is closed under union, complement, concatenation, and star.

  \begin{solution}
    Let $M_{1}$ and $M_{2}$ be 2 turing machines which decide languages $L_{1}$ and $L_{2}$ in polynomial time.
    \begin{enumerate}
      \item union\\
            Construct a turing machine $M_{U}$ which decides the union of $L_{1}$ and $L_{2}$.

            $M_{U}$ on input $w$:\\
            \null \quad 1. Run $M_{1}$ on $w$. If accept return accept.\\
            \null \quad 2. Run $M_{w}$ on $w$. If accept return accept.\\
            3. If both reject then return reject.

            Since both $M_{1}$ and $M_{2}$ decide in polynomial time then the union will also decide in polynomial time.

      \item complement\\
            Construct a deterministic turing machine $M_{\bar C}$ which decies on the complement of $L_{1}$.\\

            $M_{\bar C}$ on input $w$:\\
            \null \quad 1. Run $M_{1}$ on input $w$\\
            \null \quad 2. If $M_{1}$ accept then return reject, else return accept.

            Since $M_{1}$ runs in polynomial time then $M_{\bar C}$ also runs in polynomial time.

      \item concatenation\\
            Construct a determisnistic TM $M_{C}$ which decides the concatenation of languages $L_{1}$ and $L_{2}$ in polynomial time.\\

            $M_{C}$ on input $w$:
            \null \quad  1. Guess the partition of input $w$ into 2 string such as $w = xy$.\\
            \null \quad 2. Run $M_{1}$ on input $x$. If not accept return reject.\\
            \null \quad 3. Run $M_{2}$ on input $y$. If not accept return reject.\\
            \null \quad 4. If both accept then return accept

            Since both $M_{1}$ and $M_{2}$ run in polynomial time then $M_{C}$ also runs in polynomial time.


      \item star\\
            Construct a determisnistic TM $M_{S}$ which decides on $L_{1}^{*}$. This is similar to concatenation where we divide the input $w$ into several sub strings.

            $M_{S}$ on input $w$:\\
            \null \quad 1. Guess the partition $w = x_{1}x_{2}\dots x_{n}$.\\
            \null \quad 2. If $M_{1}$ accepts all of $x_{n}$ then return accept else return reject.

            Since $M_{1}$ runs in polynomial time then $M_{S}$ also runs in polynomial time.

    \end{enumerate}

  \end{solution}

  \question{} % q3

  Show that NP is closed under union, concatenation, and star

  \begin{solution}
    The solution is similar to Q2, instead of using a deterministic TM we would use a non-deterministic TM which would go through every branch to check if the input is accept by the TM.
  \end{solution}

  \question{} % q4

  We normally assume natural numbers are represented in binary, such that a number $n\in \mathcal{N}$ is represented by the string $b_{\lfloor\log n \rfloor}b_{\lfloor \log n \rfloor - 1} \dots b_{0}, b_{i} \in \{0,1\}$, and $n=\Sigma^{\lfloor \log n \rfloor}_{i=0}b_{i}2^{i}$.
  We could also write a number in unary, where a number $n\in \mathcal{N}$ is represented by $n$ consecutive 1s.
  The problem of factoring a number in binary is not known to be in P, but what if the number is given in unary?
  Prove your answer.

  \begin{solution}

    Lets look at a naive algorithm of finding factors of a input n.

    for i = 1 to $\sqrt n$:\\
    \null \quad if (n \% i == a):\\
    \null \quad \quad add i as a factor

    If the numbers are represented in binary then the length of the input is $k = \log_{2} n$ bits which makes the time complexity to be $\mathcal{O}(\sqrt{2^{k}})$ therby making the problem to be exponential.


    In the case of unary the lenth of the input is equal to $n$ bits which makes the time complexity to be linear. Therefor this algorithm is in $P$.

  \end{solution}

  \question{} % q5

  Suppose the number $k$ and the graph $G$ are given, and we want to know if there exists a subset $S$ of size $k$ from the vertices of $G$ such that there is no edge between them in $G$.
  Prove that this problem is NP-Complete.

  \begin{solution}

    NO-EDGE = $\{ \langle G, k\rangle : G$ is an undirected graph and has a subset S of size k with no edges between them $\}$

    Show that this problem is NP, a certificate for this problem is given as follows:

    \textbf{Proof:} certificate is a subset of vertices c\\
    \emph{V$G,k,c$}:\\
    \null \quad verify that $G$ has $k$ verticies.\\
    \null \quad verify that there are no edges between the nodes in $c$ in graph $G$\\
    \null \quad accept $\leftrightarrow$ both pass

    Show that this problem in NP-Complete

    \textbf{Proof:} We use reduction from $CLIQUE \leq_{p} NO-EDGE$

    Given a graph $G$ with $k$ clique. We can construst a graph $G'$ with $V'$ vertices and $E'$ edges.\\
    \null \quad We find the complement of the graph $G$ to produce $G'$.\\
    \null \quad Traverse through $G$ and where there is an edge between 2 vertices remove it and where there is no edge we add an edge between the pair of verticles.
    \null \quad We get the complement graph $G'$ the edges and $k$ is the number of verticles that have no edges between them.

    This runs in poly time.

  \end{solution}


  \question{} % q6

  Show the following languge is NP-Complete:
  \[DOUBLE-SAT=\{\langle \phi \rangle | \phi \text{ has at least 2 satisfying assignments }\}\]

  \begin{solution}

    Showing DOUBLE-SAT is NP is easy; a certificate is simply a set of values of the variables in $\phi$ which make the boolean expression evaluate to true.

    We will use poly time reduction to prove that DOUBLE-SAT is NP-Complete by showing $SAT \leq_{p} DOUBLE-SAT$


    \textbf{Proof:}\\
    We create a non-deterministic TM $M$ such that it computes a polytime reduction of $\phi \to \phi'$\\
    $M$ on input $\langle \phi \rangle$ which a Boolean formula such that $\phi$ has variables $c_{1}, c_{2}, \dots , c_{k}$.

    \null \quad 1. Create $\phi' = \phi \land (x \lor \bar x)$ with $x$ as a new variable.\\
    \null \quad 2. Run $\phi$ on SAT.\\
    \null \quad 3. If SAT accepts then $\phi'$ has at least 2 satisfying assignment. If $\phi \notin SAT$ then $\phi' \notin DOUBLE-SAT$

    We showed in TM $M$ that $\phi \in SAT$ iff $f(\phi)\in DOUBLE-SAT$ therefore DOUBLE-SAT is NP-Complete.

  \end{solution}

  \question{} % q7

  Let $S$ be a set and let $C$ be a collection of subsets of $S$. A set $S'\subseteq S$ is called a set hitting set for $C$ if every subset in $C$ contains at least an element in $S'$. Let
  \[HITSET = \{\langle C,k \rangle | C \text{ has a hitting set of size } k\}\]

  Prove that $HITSET$ is NP-Complete.

  \begin{solution}
    We can show that HITSET is in $NP$


    \textbf{Proof:} certificate is a certificate is simply a the set $S'$ where $S' \cap S_{i} \neq \emptyset, \forall S_{i} \in C$.

    \emph{V($C, k, c$)}:\\
    \null \quad verify that $c$ is in $C$\\
    \null \quad verify that each set in $C$ has $S_{i}\cap c \neq \emptyset$\\
    \null \quad accept $\leftrightarrow$ both pass

    runs in polytime

    Now show that $VERTEX-COVER \leq_{p} HITSET$




  \end{solution}

  \question{} % q8

  \begin{enumerate}[a.]
    \item Prove that NP $=$ coNP iff there is an NP-Complete problem in coNP.
    \item Show that if coNP $\neq$ NP then P $\neq$ NP.
  \end{enumerate}


  \begin{solution}
    \begin{enumerate}[a.]
      \item NP $=$ coNP iff there is an NP-Complete problem in coNP\\

            \textbf{Proof:}\\
            Part 1: if there is an NP-Complete problem in $coNP$ then $NP=coNP$


            Let $L$ be an language that is NP-Complete and $L\in coNP$.
            Let $L_{2}$ be a problem in NP then there is a polytime reduction $f$ from $L_{2} \to L$ since $L$ is NP-Complete.
            Now since $L_{2}\to L$ exists this shows that $L_{2}$ is also in $coNP$. So $NP=coNP$


            Part 2: if $NP=coNP$ then there is an NP-complete in coNP

            This statement is true itself since $NP=coNP$ there would be problem which are NP-complete in $coNP$


      \item if coNP $\neq$ NP then $P\neq NP$

            \textbf{Proof:}\\

            We can prove this by showing the contrapositive.\\
            If $P=NP$ then coNP$=$NP

            Let $L$ be a problem which in in $P$. Since $P=NP$ then $L$ is also in $NP$. We know that $P$ is closed under complement then $NP$ would also be closed under complement. Therefore $\bar L$ is also in $P$ and $NP$. This shows that $NP = coNP$.

    \end{enumerate}

  \end{solution}

  \question{} % q9

  Show that P is closed under homomorphism iff P $=$ NP

  \question{} % q10

  Let $CNF_{k} = \{\langle \phi \rangle | \phi \text{ is a satisfiable cnf-formula where each variable appears in at most } k \text{ places }\}$.

  \begin{enumerate}[a.]
    \item Show that $CNF_{2} \in P$
    \item Show that $CNF_{3}$ is NP-complete
  \end{enumerate}



  \begin{solution}
    \begin{enumerate}[a.]

      \item Show that $CNF_{2}\in P$

            We give a way to fine out if a variable appears more than twice in a binary formula. We say it case by case:

            Let $x$ be a variables in $\phi$\\
            \null \quad 1. If $x$ appeas only once as $x$ or $\bar x$ then we can satisfy the clause it appears in and move onto next variable.\\
            \null \quad 2. If $x$ or $\bar x$ appears more 2 times then $\phi$ is not satisfiable\\
            \null \quad 3. If $x$ or $\bar x$ appears in the same clause then $\phi$ is not satisfiable.\\
            \null \quad 4. If $x$ or $\bar x$ appears in different clauses such as $(x\lor \dots) \land (\bar x \lor \dots)$ then it satisfiable only if other variables in the clauses have an satisfying assignment as both $x$ and $\bar x$ cannot be satisfiable at the same time.

            We use this method and easily see that it reduces the boolean formula each time therefore it runs in polynomial time and is in P.


      \item Show that $CNF_{3}$ is NP-Complete.

            Check that $CNF_{3}\in NP$

            \textbf{Proof:} certificate is a satisfying argument\\
            \emph{V($\phi, c$)}:\\
            \null \quad verfiy each variables appeas at most 3 times\\
            \null \quad evaluate $\phi(c)$\\
            \null \quad accept $\leftrightarrow$ both pass

            Prove that $3-SAT\leq_{p} CNF_{3}$







    \end{enumerate}



  \end{solution}
  
\end{questions}


\end{document}
