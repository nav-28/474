\documentclass[12pt]{exam}
\printanswers
\usepackage[utf8]{inputenc}
\usepackage[a4paper, total={6in,4in}]{geometry}
\usepackage{geometry}
\usepackage{amsmath}
\usepackage{multicol}
\usepackage{amssymb}
\usepackage{enumerate}
\geometry{left=2cm, right=2cm, top=2cm, bottom=2cm}
\usepackage{tikz}
\usepackage{tikz-qtree}
\usepackage{fixltx2e}
\usepackage{tabto}
\usepackage{makecell}
\usetikzlibrary{automata, positioning, arrows}
\tikzset{
->, % makes the edges directed
>=stealth', % makes the arrow heads bold
node distance=3cm, % specifies the minimum distance between two nodes. Change if necessary.
every state/.style={thick}, % sets the properties for each ’state’ node
initial text=$ $, % sets the text that appears on the start arrow
}
\newcommand\scalemath[2]{\scalebox{#1}{\mbox{\ensuremath{\displaystyle #2}}}}



\title{
  Assignment 2\\
  \large CMPUT 474
}
\author{Pranav Wadhwa\\1629510}

\begin{document}
\maketitle
\noindent


\begin{questions}

  \question{} % question 1
  A \textbf{Turning mahine with doubly infinite tape} is similar to an ordinary Turing machine, but its tape is infinite to the left as well as to the right.
  The tape is initially filled with blanks except for the portion that contains the input.
  Computations is defined as usual except that the head never encounters an end to the tape as it moves leftward.
  Show that this type of Turing Machine recognizes the class of Turing-recognizable languages.

  \begin{solution}
    Lets first examine what happen in a regular turing machine when its head is at the left-end of the tape.
    If the head of the machine is at the left-end and it tries to do an $\delta(q,\sigma) = (q',\sigma ', L)$ but the head stays at the same place.

    We want to show that Turing Machine with doubly infinite tape recognizes the class of Turing-recognizable languages.
    We can do this by making the new TM simulate the left bounded machine by adding a new symbol to mark the left end of the regular TM's tape and prevent the head from moving any further left off the that mark.

    To simulate the doubly infinite tape by an regular TM we can use 2 tapes. As we already know that we can easily simulate multiple tapes on a single tape, we can use this knowledge to simulate the doubly infinite tape.
    The first tape would have the input symbol and the right bound and the second tape would have the left bound in reverse order. At the start of the computation the second tape would be black.
    This way we can get a regular TM to simulate a doubly infinite TM.
  \end{solution}

  \question{} % question 2
  Show that the collection of decidable languages is closed under the operation of
  \begin{multicols}{2}
  \begin{enumerate}
    \item union
    \item concatenation
    \item star
    \item complementation
    \item intersection
  \end{enumerate}
  \end{multicols}

  \begin{solution}
    Let there be 2 decidable languages $L_{1}$ and $L_{2}$ and their respective turing machines that accept then $M_{a}$ and $M_{2}$.
    As both $M_{1}$ and $M_{2}$ are decidable both machine will halt.

    \begin{enumerate}
      \item union

            We construct a turing machine $M$ that recognizes $L_{1}\cup L_{2}$ such that

            \begin{enumerate}[i]
            \item Read input string $w$
            \item  Run $M_{1}$ on string $w$.\\
                    \null \quad If $M_{w}$ return reject goto next step. Else return accept
            \item Run $M_{2}$ on string $w$.\\
                    \null \quad If $M_{2}$ return reject then return reject. Else return accept

            \end{enumerate}


      \item concatenation

            We will use the \textbf{non-deterministic} TM to prove this. The concatenation of $L_{1}$ and $ L_{2}$ is defined as $L_{1}\circ L_{2} = \{xy| x\in L_{1} \text{ and } y\in L_{2}\}$.
            Consider the non-deterministic TM $M$ which would take an input $w$ and partition it into a form $w = xy$ where $x \in L_{1}$ and $y\in L_{2}$.

            \begin{enumerate}[i]
              \item Read the input $w$ and partition it into 2 string $w = xy$
              \item Simulate $M_{1}$ on $x$ and simulate $M_{2}$ on $y$
              \item return accept if both $M_{1}$ and $M_{2}$ accept it, else return reject.
            \end{enumerate}

      \item star

      \item complementation

            We have a TM $M$ which simulates $L_{1}$. If $L_{1}$ rejects input $w$ then $M$ return accepts and it $L_{1}$ accepts input $w$ then $M$ returns rejects.

      \item intersection

            Intersection is similar to union. Let $M$ be a TM which simulates $L_{1}\cap L_{2}$  where we run an input $w$ on both machines and $M$ return accept only when both $L_{1}$ and $L_{2}$ return accept else it return reject.

    \end{enumerate}

  \end{solution}
  \question{} %question 3
  Show that the collection of Turing-recognizable languages is closed under the operation of

  \begin{multicols}{2}
  \begin{enumerate}
    \item union
    \item concatenation
    \item star
    \item intersection
    \item homomorphism
  \end{enumerate}
  \end{multicols}

  (Note that you can fine the definition of homomorphism on Page 93, Problem 1.66)


  \begin{solution}
    Let there be 2 Turing-recognizable languages $L_{1}$ and $L_{2}$ and their respcitive TM $M_{1}$ and $M_{2}$.
    \begin{enumerate}
      \item union

            We construct TM $U$ that recognizes $L_{1}\cup L_{2}$ such that:
            \begin{enumerate}[1.]
              \item Simulate $M_{1}$ and $M_{2}$ on string $w$ simultaneously.
              \item Do a step of $M_{1}$ first then do a step of $M_{2}$
              \item If any of the machine return accept then return accept.
            \end{enumerate}


      \item concatenation
            This is similar to concatenation of decidable languages. Let a non-deterministic TM simulate $M_{1}$ and $M_{2}$ on a string $w$ which can be partitoned into $w=xy$ where $x\in L_{1}$ and $y\in L_{2}$.
            We simulate $M_{1}$ on $x$ and $M_{2}$ on $y$ simulataneouly. If both accept then return accept.

      \item star

      \item intersection

            This operation is similar to the union of decidable languages. Let a TM $I$ simulate input $w$ on $M_{1}$ then on $M_{2}$. If both return accept then return accept else return reject.

      \item homomorphism

            \textbf{Def:} A \emph{homomorphism} is a function $f:\Sigma\to \Gamma^{*}$ from one alphabet to strings over another alphabet.


    \end{enumerate}


  \end{solution}


  \question{} % question 4
  Prove the following language is decidable

  \[L=\{\langle M \rangle :M\text{ is a DFA that accepts some string of the form } ww^{R} \text{ for } w\in \{0,1\}^{*}\}\]

  \begin{solution}

    The string of the form $ww^{R}$ for $w\in \{0,1\}^{*}$ is not a regular language but it is a context-free language. Let $A = \{ x| \text{ x is of form } w\in \{0,1\}^{*} \}$.

    It is already known that context-free langugaes are decidable. We need to show that $M$ accept string of form $A$ is decidable. We can contruct a decider TM $D_{M}$ which decides on $L$.\\
    $D_{M}$ on input $<M>$ where $M$ is a DFA:
    \begin{enumerate}
      \item Construct $B=A\cap L(M)$. $B$ is a CFL as $A$ is a CFL. (This is done through another TM which converts the DFA into a RE and does the intersection with $A$)
      \item If $B$ is empty return reject else return accept.
    \end{enumerate}

    Therefore, $L$ is decidable.
  \end{solution}

  \question{} %question 5
  Show that the following langauges are undecidable

  \begin{enumerate}[a)]
    \item Set of descriptions of Turing machines $\langle M \rangle$ such that $\emptyset \in L(M)$

    \item Set of decriptions of pairs of Turing machines $\langle M, M' \rangle$ such that $L(M)\cap L(M') = \emptyset$.
          Also show that this language is recognizable.

  \end{enumerate}


  \begin{solution}

    \begin{enumerate}[a)]
      \item Let $L=\{\langle M\rangle| M\in TM \text{ and } \emptyset \in L(M)\}$ and we show that $L$ is undecidable.

            Assume the contrary that $L$ is decidable and TM $M_{L}$ can decide on $L$. We show that $M_{L}$ can be used to decide $A_{TM}$. Then we construct TM $D$ such that:

            $D(\langle M\rangle)$
            \begin{enumerate}[1.]

              \item call $M_{L}(\langle M \rangle)$
              \item if $M_{L}$ accepts $\langle M \rangle$\\
                    \null\quad Simulate $M$ on string $x$ (must halt)\\
                    \null \quad if $M$ accepts, return accept, else return reject
              \item If $M_{L}$ rejects $\langle M\rangle$ return reject.
            \end{enumerate}

            This contradicts the undecidability of $A_{TM}$ so $L$ is undecidable.


      \item We have 2 TM $M$ and $M'$ which simulate the languages $L(M)$ and $L(M')$. We know that decidable languages are closed under intersection then $L(M)\cap L(M')$ is also a decidable language which must have an equivalent TM say $I$.
            Let $L=\{\langle M,M'\rangle| L(M)\cap L(M') = \emptyset\} = \{\langle I\rangle| I\in TM \text{ and } L(I) = \emptyset\}$

            We know that $E_{TM}$ is undecidable and $L$ is also of the same form. Therefore, this language $L$ is also undecidable.
    \end{enumerate}


  \end{solution}


  \question{} %question 6
  Let $A$ and $B$ be two disjoint languages. Say that language $C$ separates $A$ and $B$ if $A\subseteq C$ and $B\subset \bar C$. Show that any two disjoint co-recognizable languages are separable by some decidable language.


  \question{} %question 7
  Say that a variable $A$ in CFL $G$ is usable if it appears in some derivation of some strings $w\in G$. Given a CFG $G$ and a variable $A$, consider the problem of testing whether $A$ is usable.
  Formulae this problem as a language and show that it is decidable.

  \begin{solution}
    Let there be a CFG $G$ which has a variable $A$. We define that a variable is usable if it appears in some derivation of some strings. Or, more precisely, $A$ appears in the the rule of some other variable such as $X$ such that $X\to X_{1}AX_{2}$ is a rule in CFG $G$.
    Here $X_{1}$ and $X_{2}$ can be $\epsilon$ or derive other languages $L(X_{1})$ or $L(X_{2})$.

    Consider the language $U = \{\langle G,A\rangle| G \text{ is a CFG }, A \text{ is usable for G }\}$. We show that $U$ is decidable.
    Let $D$ be a decider TM which decides on $U$ and it works such that

    On Input of $\langle G,A\rangle$:
    \begin{enumerate}
      \item Let $S$ be a starting variable. It uses breath first search to look for any derivatation from $S$ to $A$. If there are no derivation to $A$ it returns reject else move to next step.
      \item We check using $E_{CFG}$ that $L(A)$ is not empty. If $E_{CFG}$ returns reject then then $D$ returns accept else $D$ returns reject.
    \end{enumerate}

    Therefore, $U$ is decidable

  \end{solution}

  \question{} %question 8
  Let $L$ be a CFL. Is $\{1\}^{*}\subseteq L$ a decidable problem?
  Is $\{1\}^{*}=L$ a decidable problem?

  \begin{solution}
    We know that $\{1\}^{*}$ is a regular language and the intersection of a regular language and a CFL is a CFL.

    We will use this knowledge to construct a decider TM $D$ such that:

    On input of a CFL $L$
    \begin{enumerate}[1.]
      \item take the intersection: $L\cap \{1\}^{*} = A$
      \item Check that if $A=\emptyset$
      \item If A is empty then return false else return true.
    \end{enumerate}

    So $\{1\}^{*}\subseteq L$ is a decidable problem.


    Now to check if $\{1\}^{*} = L$ is decidable problem.
    We can't take the intersection like the previous problem as $EQ_{CFG}$ is undecidable.
    Instead,we will look at how the CFG generates it strings. The idea is to look at all the rules in the CFG in the chomsky normal form and check if the terminal variables are all $1s$ or not.

    So let a TM $D$ be a decider such that:

    On input of a CFL $L$:
    \begin{enumerate}
      \item Convert the CFL into chomsky normal form.
      \item Look at all the terminal variables of the CFG.
      \item If there is a production rule (other than the start variable) such
            that it has a terminal symbol other than 1, return reject.
      \item if all production rules with a terminal symbols are 1, then return accept.
    \end{enumerate}

    So, $\{1\}^{*} = L$ is also a decidable problem.


  \end{solution}

  \question{}
  Consider the problem of determining whether a turing machine $M$ on an input $w$ ever attempts to move its head left at any point during its computation on $w$. Formulate this problem as a language and show that it is decidable.

  \begin{solution}

    let $L=\{\langle M,w\rangle| M \text{ moves its head left on input } w\}$. We show that $L$ is decidable using a TM $LM$.

    We construct TM $LM$ such that it simulates $M$ on input $w$. We will have a few cases during the execution on $w$ such as:

    \begin{enumerate}[a)]
      \item If the head ever moves left then return accept.
      \item If the head never moves left and accepts the input.
      \item if the head stays in the same position during the execution after reading through the input and going over $|Q|$ (number of states) steps then return reject.
      \item if the head moves right and have gone over the input and $|Q|$ (number of states) steps we can conclude that $M$ is in a cycle and thereby never moving left.
    \end{enumerate}

    We can formulate this into an high level description of the machine $LM$ such that if $M$ moves left withing $w+|Q|$ steps then return accept else return reject as after that steps the machine enters into a cycle.

    Therefore, the language $L$ is decidable.



    
  \end{solution}

  \question{}
  Consider the problem of determining whether a Turing machine $M$ on an input $w$ ever
  attempts to move its head left when its head is in the left-most tape cell.
  Formulate this problem as a language and show that it is undecidable.


  \begin{solution}

    Let $L = \{\langle M,w\rangle| \text{head attempts to move its head left when head is in the left-move tape cell }\}$. We show that $L$ is undecidable using reduction of $HALT_{TM}$.

    Let $LM$ be the TM that run $M$ on input $w$ but puts a $\#$ symbol on the left end of the tape and shifts the input after $\#$ symbol.

    Let $LM'$ be the decider TM and it run on input $\langle M,w\rangle$ as follows:
    \begin{enumerate}[1.]
      \item Run TM $LM$ to simulate $M$ on input $w$ as described above.
      \item If $LM$ reaches the $\#$ symbol during its execution, then it moves it head one step to the right but remains in the same state, then it return accept, else return reject.
      \item if $LM$ accept\\
            Simulate $M$ on $w$ (must halt)
            if $M$ return accept, return accept else return accept
      \item If $LM$ rejects $M$ and $w$\\
            then return reject
    \end{enumerate}

    This contradicts the undecidability of $HALT_{TM}$ so $L$ is also undecidable.

  \end{solution}

\end{questions}



\end{document}
