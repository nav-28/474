\documentclass[12pt]{exam}
\printanswers
\usepackage[utf8]{inputenc}
\usepackage[a4paper, total={6in,4in}]{geometry}
\usepackage{geometry}
\usepackage{amsmath}
\usepackage{multicol}
\usepackage{amssymb}
\geometry{left=2cm, right=2cm, top=2cm, bottom=2cm}
\usepackage{tikz}
\usepackage{tikz-qtree}
\usepackage{fixltx2e}
\usepackage{makecell}
\usetikzlibrary{automata, positioning, arrows}
\tikzset{
->, % makes the edges directed
>=stealth', % makes the arrow heads bold
node distance=3cm, % specifies the minimum distance between two nodes. Change if necessary.
every state/.style={thick}, % sets the properties for each ’state’ node
initial text=$ $, % sets the text that appears on the start arrow
}
\newcommand\scalemath[2]{\scalebox{#1}{\mbox{\ensuremath{\displaystyle #2}}}}



\title{
  Assignment 2\\
  \large CMPUT 474
}
\author{Pranav Wadhwa\\1629510}

\begin{document}
\maketitle
\noindent


\begin{questions}

  \question{}
  A \textbf{Turning mahine with doubly infinite tape} is similar to an ordinary Turing machine, but its tape is infinite to the left as well as to the right.
  The tape is initially filled with blanks except for the portion that contains the input.
  Computations is defined as usual except that the head never encounters an end to the tape as it moves leftward.
  Show that this type of Turing Machine recognizes the class of Turing-recognizable languages.



  \question{}
  Show that the collection of decidable languages is closed under the operation of
  \begin{multicols}{2}
  \begin{enumerate}
    \item union
    \item concatenation
    \item star
    \item complementation
    \item intersection
  \end{enumerate}
  \end{multicols}


  \question{}
  Show that the collection of Turing-recognizable languages is closed under the operation of

  \begin{multicols}{2}
  \begin{enumerate}
    \item union
    \item concatenation
    \item star
    \item intersection
    \item homomorphism
  \end{enumerate}
  \end{multicols}

  (Note that you can fine the definition of homomorphism on Page 93, Problem 1.66)


  \question{}
  Prove the following language is decidable

  \[L=\{\langle M \rangle :M\text{ is a DFA that accepts some string of the form } ww^{R} \text{ for } w\in \{0,1\}^{*}\}\]


  \question{}
  Show that the following langauges are undecidable

  \begin{enumerate}
    \item Set of descriptions of Turing machines $\langle M \rangle$ such that $\emptyset \in L(M)$

    \item Set of decriptions of pairs of Turing machines $\langle M, M' \rangle$ such that $L(M)\cap L(M') = \emptyset$.
          Also show that this language is recognizable.

  \end{enumerate}


  \question{}
  Let $A$ and $B$ be two disjoint languages. Say that language $C$ separates $A$ and $B$ if $A\subseteq C$ and $B\subset \bar C$. Show that any two disjoint co-recognizable languages are separable by some decidable language.


  \question{}
  Say that a variable $A$ in CFL $G$ is usable if it appears ni some derivation of some strings $w\in G$. Given a CFG $G$ and a variable $A$, consider the problem of testing whether $A$ is usable.
  Formulae this problem as a language and show that it is decidable.


  \question{}
  Let $L$ be a CFL. Is $\{1\}^{*}\subseteq L$ a decidable problem?
  Is $\{1\}^{*}=L$ a decidable problem?

  \question{}
  Consider the problem of determining whether a turing machine $M$ on an input $w$ ever attempts to move its head left at any point during its computation on $w$. Formulate this problem as a language and show that it is decidable.


  \question{}
  Consider the problem of determining whether a Turing machine $M$ on an input $w$ ever
  attempts to move its head left when its head is in the left-most tape cell.
  Formulate this problem as a language and show that it is undecidable.

\end{questions}



\end{document}
